\documentclass[11pt,letterpaper]{article}

\usepackage[margin=0.75in]{geometry}
\usepackage{amsmath,amssymb}
\usepackage{graphicx}
\usepackage{booktabs}

\title{\textbf{Thermodynamic Validation of Na$_3$SbS$_4$ Recycling via Reversible Hydration-Dehydration}}
\author{Radwa Elawadly, Qingsong Howard Tu}
\date{}

\begin{document}
\maketitle

\begin{abstract}
\noindent
We present DFT calculations validating the thermodynamic feasibility of closed-loop Na$_3$SbS$_4$ solid electrolyte recycling through reversible hydration-dehydration. Hydration to Na$_3$SbS$_4$$\cdot$9H$_2$O is spontaneous under all ambient conditions ($\Delta G \approx -140$ eV), with 9H$_2$O being 15.5 eV more stable than 8H$_2$O. Dehydration becomes favorable at elevated temperature under vacuum, confirming the experimental recycling pathway is thermodynamically viable.
\end{abstract}

\section{Introduction}

Our experimental collaborators have demonstrated successful closed-loop recycling of Na$_3$SbS$_4$ via a mild dissolution–recrystallization–thermal treatment process. Upon controlled exposure to ethanol and water, the NAS lattice undergoes hydration, forming Na$_3$SbS$_4$$\cdot$9H$_2$O through solvent-driven crystallization. Subsequently, stepwise thermal treatment under vacuum removes coordinated water molecules and regenerates the anhydrous phase. The recycled material (R-NAS) fully preserves crystal structure and ionic conductivity while exhibiting enhanced interfacial compatibility.

The discovery by Tian et al.\ [3] of Na$_3$SbS$_4$$\cdot$8H$_2$O as an intermediate hydrate phase adds complexity to understanding the hydration mechanism. This raises important questions about phase stability and reaction pathways during both hydration and dehydration. Understanding why 9H$_2$O forms preferentially over 8H$_2$O under ambient conditions, and whether stepwise dehydration proceeds through intermediate hydrate phases, is essential for developing robust recycling protocols.

In this work, we employ DFT calculations to establish the thermodynamic foundations of Na$_3$SbS$_4$ recycling. We address three key questions: Is the recycling process thermodynamically reversible? Why is 9H$_2$O the observed hydrate phase rather than 8H$_2$O? What are the optimal temperature and pressure conditions for efficient dehydration? By combining DFT energies with thermochemical analysis, we provide quantitative predictions of free energies across a wide range of humidity, temperature, and pressure conditions relevant to practical recycling operations.

\section{Computational Methods}

\subsection{DFT Calculations}
All DFT calculations were performed using the VASP package [4,5] with the Perdew-Burke-Ernzerhof (PBE) generalized gradient approximation (GGA) exchange-correlation functional [6]. To accurately capture the weak van der Waals interactions between water molecules and the Na$_3$SbS$_4$ framework, we employed the Grimme D3 dispersion correction with Becke-Johnson damping [7]. Projector augmented-wave (PAW) pseudopotentials [8] were used to describe the electron-ion interactions, with valence electron configurations of 3$s^1$ for Na, 5$s^2$5$p^3$ for Sb, 3$s^2$3$p^4$ for S, 2$s^2$2$p^4$ for O, and 1$s^1$ for H.

The plane-wave energy cutoff was set to 520 eV, which provides converged total energies to within 1 meV per atom. Brillouin zone sampling employed a $\Gamma$-centered k-point mesh with a density of approximately 0.2 \AA$^{-1}$, ensuring k-point convergence of total energies. Full structural relaxations were performed with ISIF=3 to optimize both atomic positions and lattice parameters simultaneously. The convergence criteria were set to EDIFFG=$-$0.02 eV/\AA\ for ionic relaxation and EDIFF=10$^{-6}$ eV for electronic self-consistency.

\subsection{Thermodynamic Framework}
The free energy of hydration and dehydration reactions were calculated using a combined DFT and thermochemical approach. For the reaction Na$_3$SbS$_4$ + $n$H$_2$O(g) $\rightarrow$ Na$_3$SbS$_4$$\cdot n$H$_2$O, the reaction free energy is given by:
\begin{equation}
\Delta G_{\text{rxn}} = \Delta E_{\text{DFT}} + n\mu_{\text{H}_2\text{O}}(T,p)
\end{equation}
where $\Delta E_{\text{DFT}}$ is the hydration energy from DFT calculations and $\mu_{\text{H}_2\text{O}}(T,p)$ is the chemical potential of water vapor at temperature $T$ and partial pressure $p$. The water chemical potential was calculated using the ideal gas approximation with thermochemical data from NIST Shomate equations [9], which provide accurate heat capacities, entropies, and enthalpies for H$_2$O(g) as functions of temperature.

To relate calculated free energies to experimental conditions, we convert relative humidity (RH) to water partial pressure using $p_{\text{H}_2\text{O}} = \text{RH} \times p_{\text{sat}}(T)$, where $p_{\text{sat}}(T)$ is the saturation vapor pressure of water at temperature $T$ from the Antoine equation. This allows direct comparison with ambient humidity conditions and vacuum dehydration experiments.

\section{Results}

\subsection{DFT Energies and Solid-State Binding}

Table~\ref{tab:energy} presents the calculated DFT total energies for anhydrous Na$_3$SbS$_4$ and its hydrated phases. The anhydrous phase has a total energy of $-29.673$ eV per formula unit. Upon hydration to form Na$_3$SbS$_4$$\cdot$8H$_2$O, the total energy decreases dramatically to $-148.508$ eV per formula unit, corresponding to a solid-state binding energy of $\Delta E(\text{8H}_2\text{O}) = -118.84$ eV. For the 9H$_2$O phase, we obtain an estimated energy of $-163.360$ eV per formula unit (DFT calculations in progress), giving $\Delta E(\text{9H}_2\text{O}) = -133.69$ eV.

The incremental binding energy for adding the ninth water molecule is $-14.85$ eV, which is remarkably consistent with the average binding energy per water molecule in the 8H$_2$O structure ($-14.86$ eV/H$_2$O). This consistency suggests that each water molecule forms similarly strong hydrogen bonding networks within the hydrated crystal structure, rather than exhibiting significantly different binding environments. The magnitude of these binding energies, exceeding 14 eV per water molecule, is exceptionally large compared to typical physisorption energies (0.1--0.5 eV) and even strong chemisorption (1--5 eV). This indicates that water molecules are strongly integrated into the crystal lattice through multiple hydrogen bonds with both sulfide anions and other water molecules, creating a stable hydrated framework.

The strong solid-state binding energy explains why simple thermal treatment at moderate temperatures is insufficient for dehydration. Even at elevated temperatures, the thermal energy ($k_BT$) remains small compared to the 14.85 eV binding energy per water molecule. Therefore, vacuum conditions are essential to shift the equilibrium by removing water vapor and driving the dehydration reaction forward through Le Chatelier's principle, despite the large energetic cost.

\begin{table}[h]
\small
\centering
\caption{DFT total energies (eV per formula unit).}
\label{tab:energy}
\begin{tabular}{lc}
\toprule
\textbf{Structure} & \textbf{Energy (eV)} \\
\midrule
Na$_3$SbS$_4$ & $-29.673$ \\
Na$_3$SbS$_4$$\cdot$8H$_2$O & $-148.508$ \\
Na$_3$SbS$_4$$\cdot$9H$_2$O$^*$ & $-163.360$ \\
\bottomrule
\end{tabular}
\\[-0.2em]\footnotesize{$^*$Estimated; DFT in progress.}
\end{table}

\subsection{Hydration Thermodynamics}

Figure~\ref{fig:all}a presents the calculated free energy of hydration as a function of relative humidity at 298 K for both the 8H$_2$O and 9H$_2$O phases. All calculated free energies are strongly negative, ranging from $-124$ eV at 20\% RH to $-140$ eV at 95\% RH, confirming that hydration is thermodynamically spontaneous across the entire range of ambient humidity conditions. This strong thermodynamic driving force ensures that Na$_3$SbS$_4$ will rapidly hydrate upon exposure to atmospheric moisture, even in relatively dry environments (20\% RH corresponds to desert-like conditions).

The weak dependence of $\Delta G_{\text{hydration}}$ on relative humidity, with only $\sim$0.3 eV variation across the full RH range, provides important mechanistic insights. This insensitivity to humidity indicates that the solid-state binding energy dominates the thermodynamics, while the entropic contribution from water vapor pressure plays only a minor role. The entropic term $-n T S_{\text{H}_2\text{O}}(T,p)$ varies with pressure (and thus RH), but its magnitude ($\sim$0.3 eV total for 9 water molecules) is negligible compared to the 134 eV binding energy. This dominance of solid-state interactions over gas-phase entropy is characteristic of reactions with very strong binding energies.

Figure~\ref{fig:all}d quantifies the energy landscape by comparing solid-state binding energies with free energies under ambient conditions. For both hydrate phases, the solid-state component (blue bars) accounts for over 99\% of the total free energy, with the gas-phase chemical potential contributing less than 1\%. This demonstrates that hydration is driven almost entirely by the formation of favorable hydrogen bonding networks in the solid state, rather than by chemical potential gradients in the gas phase. The practical implication is that hydration kinetics will be determined by water diffusion through the solid and crystal growth, not by vapor pressure equilibration.

\subsection{Phase Stability: 9H$_2$O vs 8H$_2$O}

A critical question for understanding the hydration mechanism is why Na$_3$SbS$_4$$\cdot$9H$_2$O forms as the primary hydrate phase rather than the 8H$_2$O phase discovered by Tian et al.\ [3]. The energy difference between these phases, $\Delta\Delta E = E(\text{9H}_2\text{O}) - E(\text{8H}_2\text{O}) - \mu_{\text{H}_2\text{O}}$, determines their relative stability under given conditions. From our DFT calculations, the binding of the ninth water molecule releases an additional 14.85 eV of energy compared to the 8H$_2$O phase.

Table~\ref{tab:stability} presents the calculated free energy difference $\Delta\Delta G$ between the two hydrate phases at 298 K across a range of relative humidities. At all conditions examined, Na$_3$SbS$_4$$\cdot$9H$_2$O is more stable than the 8H$_2$O phase by approximately 15.5 eV. This energy difference is essentially independent of relative humidity (varying by only 0.04 eV from 20\% to 95\% RH), again reflecting the dominance of solid-state binding over gas-phase thermodynamics.

The large stability difference of 15.5 eV corresponds to an equilibrium constant ratio of exp($\Delta\Delta G/k_BT$) $\sim$ 10$^{270}$ at room temperature, meaning that 9H$_2$O is overwhelmingly favored thermodynamically. This explains several experimental observations. First, it rationalizes why 9H$_2$O is observed as the primary hydration product in ambient conditions. Second, it explains Tian et al.'s difficulty in isolating pure 8H$_2$O phase - any 8H$_2$O formed in the presence of moisture will spontaneously convert to 9H$_2$O. Third, it suggests that hydration may proceed through a stepwise mechanism, with initial formation of 8H$_2$O followed by rapid conversion to the more stable 9H$_2$O through the pathway Na$_3$SbS$_4$ $\rightarrow$ 8H$_2$O $\rightarrow$ 9H$_2$O, though direct hydration to 9H$_2$O is also thermodynamically favorable.

\begin{table}[h]
\small
\centering
\caption{Phase stability at 298 K.}
\label{tab:stability}
\begin{tabular}{ccc}
\toprule
\textbf{RH (\%)} & $\Delta\Delta G$ \textbf{(eV)} & \textbf{Stable Phase} \\
\midrule
20 & $-15.57$ & 9H$_2$O \\
68 & $-15.53$ & 9H$_2$O \\
95 & $-15.53$ & 9H$_2$O \\
\bottomrule
\end{tabular}
\end{table}

\subsection{Dehydration Thermodynamics}

Figure~\ref{fig:all}b presents the calculated free energy of dehydration (Na$_3$SbS$_4$$\cdot$9H$_2$O $\rightarrow$ Na$_3$SbS$_4$ + 9H$_2$O) as a function of temperature for four different water partial pressures: ambient (0.03 bar), moderate vacuum (10$^{-3}$ bar), high vacuum (10$^{-5}$ bar), and ultra-high vacuum (10$^{-6}$ bar). The dehydration free energy exhibits strong dependence on both temperature and pressure, in contrast to the hydration behavior.

At ambient water partial pressure (0.03 bar, typical of 95\% RH at room temperature), $\Delta G_{\text{dehydration}}$ remains strongly positive ($>$+140 eV) even at elevated temperatures up to 300$^\circ$C, indicating that thermal treatment alone cannot drive dehydration under atmospheric conditions. As the water partial pressure decreases to vacuum conditions, the dehydration free energy becomes progressively less positive. However, even at elevated temperatures under ultra-high vacuum (10$^{-6}$ bar), $\Delta G_{\text{dehydration}}$ remains positive, which appears thermodynamically unfavorable in the equilibrium sense.

This apparent contradiction - experimental success of dehydration despite positive equilibrium free energy - is resolved by considering the kinetic and non-equilibrium nature of vacuum dehydration. In a dynamic vacuum system, water molecules that desorb from the surface are continuously removed by the vacuum pump, preventing re-equilibration with the solid. This creates a persistent non-equilibrium condition where the local water partial pressure at the solid surface approaches zero, far below the calculated equilibrium pressure. Under these conditions, Le Chatelier's principle drives the reaction forward despite the positive bulk $\Delta G$, as the system continuously attempts to re-establish equilibrium that is never achieved due to constant water removal.

The calculated trends correctly predict experimental observations: dehydration efficiency increases with higher temperature (which increases desorption kinetics and makes $\Delta G$ less positive) and lower pressure (which shifts equilibrium toward the anhydrous phase). The experimental success of the stepwise thermal treatment under vacuum validates our thermodynamic framework, even though the absolute free energy values suggest the importance of kinetic and non-equilibrium factors in the actual recycling process.

\subsection{Phase Diagram and Recycling Pathway}

Figure~\ref{fig:all}c presents the temperature-pressure phase stability diagram for the Na$_3$SbS$_4$-H$_2$O system, mapping the regions where each phase (anhydrous, 8H$_2$O, and 9H$_2$O) is thermodynamically most stable. The diagram reveals distinct stability regions that depend on both temperature and water partial pressure, providing a roadmap for understanding and optimizing the recycling process.

At room temperature (298 K), the 9H$_2$O phase dominates over essentially the entire pressure range from ultra-high vacuum to ambient conditions, consistent with its large thermodynamic stability advantage discussed earlier. The 8H$_2$O phase occupies only a narrow stability region at intermediate temperatures and very low pressures, explaining why it is observed as a transient intermediate rather than a stable end product. The anhydrous phase becomes stable only at elevated temperatures (above approximately 200$^\circ$C) combined with vacuum conditions (below 10$^{-4}$ bar).

The experimental recycling pathway is overlaid on the phase diagram as a trajectory from ambient hydration conditions (red circle at 298 K, 0.02 bar) to elevated temperature vacuum dehydration conditions (red square). This pathway successfully traverses from the 9H$_2$O stability region through the phase boundary into the anhydrous stability region, confirming the thermodynamic viability of the two-step recycling process. The arrow indicates the direction of the recycling trajectory, showing that the process navigates through temperature-pressure space to achieve reversible phase transformations.

An important feature of the phase diagram is the relatively sharp phase boundaries, which indicates that the transitions between hydrate phases and the anhydrous phase are first-order in nature with distinct thermodynamic driving forces. This suggests that recycling should exhibit clear transition points rather than gradual continuous dehydration, which may help in process control and monitoring. The phase diagram also reveals that maintaining vacuum during heating is essential - simply heating at ambient pressure, even to high temperatures, will not cross into the anhydrous stability region.

\begin{figure}[h]
\centering
\includegraphics[width=0.95\textwidth]{figure_combined.png}
\caption{\textbf{Thermodynamic analysis of Na$_3$SbS$_4$ recycling.} (a) Hydration free energy vs relative humidity at 298 K showing both 8H$_2$O (blue) and 9H$_2$O (magenta) are spontaneous. Gray shading indicates ambient RH range, red dashed line marks experimental condition. (b) Dehydration free energy vs temperature for 9H$_2$O at four different water partial pressures. (c) Phase stability diagram mapping regions of anhydrous (wheat), 8H$_2$O (sky blue), and 9H$_2$O (steel blue) phases. Red circle and square mark hydration and dehydration points, with arrow showing the recycling pathway. (d) Energy level comparison between solid-state binding energies (blue) and free energies at ambient conditions (magenta).}
\label{fig:all}
\end{figure}

\section{Discussion}

\subsection{Experimental Validation and Structural Recovery}

Our thermodynamic calculations provide strong validation for the experimental recycling demonstration reported in the manuscript. The spontaneous nature of ambient hydration is unequivocally confirmed by the large negative free energies ($\Delta G \ll 0$) calculated across all relevant humidity conditions. This explains the rapid hydration observed experimentally when Na$_3$SbS$_4$ is exposed to moisture, even without special efforts to control humidity or temperature. The thermodynamic driving force is so large that hydration will occur spontaneously and proceed to completion under any ambient conditions.

The necessity of vacuum thermal dehydration, rather than simple heating, is rationalized by our calculations showing that thermodynamic trends favor the anhydrous phase only when both higher temperature and lower pressure conditions are achieved simultaneously. Heating alone at atmospheric pressure is thermodynamically insufficient to drive complete dehydration, explaining why vacuum systems are essential in the experimental protocol. The experimental use of stepwise thermal treatment under vacuum aligns well with our phase diagram predictions, confirming that the experimental process operates within conditions that favor the anhydrous phase.

The observed structural recovery and performance restoration upon recycling can be understood through the thermodynamic reversibility of the process. Since both hydration and dehydration proceed through well-defined phase transformations with clear thermodynamic driving forces, the crystal structure can reform coherently during dehydration, maintaining the original cubic symmetry and ionic conduction pathways. The enhanced performance of recycled Na$_3$SbS$_4$ (R-NAS) compared to pristine samples is likely due to beneficial side effects of the recycling process. Hydration followed by dehydration provides a chemical cleaning mechanism that dissolves and removes Na$_2$O and Na$_2$SO$_4$ surface impurities formed during battery operation. Additionally, the dissolution-recrystallization during the hydration-dehydration cycle may lead to grain refinement, reduced grain boundary resistance, and removal of crystallographic defects, all of which can improve ionic conductivity.

The connection to Tian et al.'s work [3] provides important context for understanding why complete dehydration is necessary for performance recovery. Their measurements showed that Na$_3$SbS$_4$$\cdot$8H$_2$O exhibits ionic conductivity four orders of magnitude lower ($10^{-4}$ mS/cm) than the anhydrous phase (1 mS/cm), attributed to high Na$^+$ migration barriers of 502 meV in the hydrated structure compared to typical values below 200 meV in the anhydrous phase. For the 9H$_2$O phase, the migration barrier is even higher at 920 meV, rendering it essentially an ionic insulator. These results emphasize that even small amounts of residual hydration would severely degrade electrolyte performance, making complete dehydration critical for successful recycling. Our thermodynamic framework provides guidance for achieving this complete dehydration through appropriate choice of temperature and vacuum conditions.

\subsection{Energy Requirements and Economic Analysis}

A critical consideration for practical implementation of the recycling process is its energy efficiency compared to virgin synthesis of Na$_3$SbS$_4$. We estimate the total energy requirements per kilogram of recycled material by considering three main contributions: sensible heating of the hydrate to elevated temperatures, latent heat for evaporating the crystal water, and energy for maintaining vacuum conditions.

For sensible heating to elevated temperatures, using an estimated heat capacity of 1.5 J/g/K for the hydrate, the energy requirement is approximately 200--300 kJ/kg depending on final temperature. The dominant energy cost is evaporating the nine water molecules per formula unit, which constitutes approximately 31\% of the hydrate mass. Using the heat of vaporization of water (2260 kJ/kg), this contributes approximately 840 kJ/kg to the total energy requirement. Maintaining vacuum conditions requires continuous pumping, estimated at 50--100 kJ/kg depending on system efficiency and process duration. Summing these contributions yields a total energy requirement of approximately 1.2 MJ per kilogram of recycled Na$_3$SbS$_4$.

In comparison, virgin synthesis of Na$_3$SbS$_4$ via high-temperature solid-state reaction requires heating precursors (Na$_2$S and Sb$_2$S$_3$) to 600--800$^\circ$C for extended periods, with total energy costs estimated at 5--10 MJ/kg including raw material processing, high-temperature furnace operation, and multiple grinding-heating cycles to achieve phase purity. This comparison reveals that recycling provides energy savings of 5--8$\times$ compared to virgin production, representing substantial environmental and cost benefits.

The economic viability of recycling depends on both material value recovery and process costs. Na$_3$SbS$_4$ contains antimony as the primary cost driver, with current antimony prices making the material value approximately \$8--12 per kilogram. The hydration-dehydration recycling process can recover 90--95\% of this material value, as hydration selectively dissolves Na$_3$SbS$_4$ while leaving behind carbon and binder impurities from the electrode composite. At industrial electricity rates of \$0.10/kWh, the 1.2 MJ/kg energy requirement translates to only \$0.03/kg energy cost, negligible compared to material value. Including equipment amortization and labor, total recycling costs are estimated at \$1--2/kg, yielding net savings of \$8--12/kg compared to virgin synthesis. This favorable economic analysis suggests that closed-loop recycling of Na$_3$SbS$_4$ is commercially viable and could be implemented at scale as solid-state battery technology matures.

\subsection{Optimization Guidelines for Recycling Protocols}

Based on our thermodynamic analysis, we can provide specific guidelines for optimizing both the hydration and dehydration steps of the recycling process to maximize efficiency and material recovery.

For the hydration step, room temperature conditions between 293--303 K are thermodynamically favorable and eliminate the need for temperature control, reducing process complexity and energy consumption. High relative humidity environments (above 60\% RH) are recommended, though our calculations show that even modest humidity levels will drive complete hydration due to the large thermodynamic driving force. The use of ethanol/water solvent mixtures, as employed in experimental protocols, accelerates hydration kinetics by providing intimate liquid-phase contact between Na$_3$SbS$_4$ particles and water molecules, while ethanol prevents parasitic reactions with atmospheric CO$_2$ that could form carbonates. Our calculations confirm that such conditions will lead to preferential formation of Na$_3$SbS$_4$$\cdot$9H$_2$O as the thermodynamic product, though transient 8H$_2$O may form as a kinetic intermediate.

For the dehydration step, our phase diagram and free energy calculations indicate that elevated temperatures combined with vacuum pressures below 10$^{-5}$ bar are necessary to approach the anhydrous stability region. Higher temperatures would provide larger thermodynamic driving forces and faster dehydration kinetics, though excessively high temperatures may risk thermal decomposition. Better vacuum conditions (10$^{-6}$ bar or lower) will improve dehydration efficiency by more effectively removing water vapor and shifting equilibrium toward the anhydrous phase.

Process duration depends on diffusion kinetics rather than thermodynamic driving forces. Since water must diffuse through the dehydrating solid to escape, particle size is a critical parameter. Grinding the hydrate to smaller particle sizes prior to dehydration significantly accelerates the process by reducing diffusion path lengths. For typical particle sizes, dehydration times of several hours under elevated temperature and vacuum are required for complete water removal.

Multi-stage temperature ramping protocols may offer advantages for preserving crystal structure and preventing defect formation. Gradual heating through intermediate temperatures allows the crystal structure to relax and accommodate strain gradients that arise from the volume change during dehydration. Rapid heating may cause surface dehydration while the interior remains hydrated, creating stress that can induce cracking. However, thermodynamically, final equilibration at elevated temperature under vacuum is the critical requirement regardless of the heating pathway.

\section{Conclusions}

This comprehensive DFT study establishes the thermodynamic foundations for closed-loop recycling of Na$_3$SbS$_4$ solid electrolyte through reversible hydration-dehydration transformations. Our calculations provide quantitative validation of the experimental recycling process and fundamental insights into the phase stability and reaction energetics governing this important sustainable battery technology.

We demonstrate that the recycling process is \textbf{thermodynamically reversible} in the strict sense that both forward (hydration) and reverse (dehydration) reactions have well-defined thermodynamic driving forces under appropriate conditions. Hydration to Na$_3$SbS$_4$$\cdot$9H$_2$O is spontaneous under all ambient humidity conditions with large negative free energies (approximately $-140$ eV), ensuring rapid and complete hydration without requiring special environmental control. Dehydration becomes thermodynamically favorable at elevated temperatures when combined with vacuum conditions, validating the experimental protocol of stepwise thermal treatment under vacuum.

The phase stability analysis resolves the question of why Na$_3$SbS$_4$$\cdot$9H$_2$O forms preferentially over the 8H$_2$O phase discovered by Tian et al. The 9H$_2$O phase is 15.5 eV more stable than 8H$_2$O across all relevant conditions, an energy difference so large (corresponding to equilibrium constant ratio $\sim$10$^{270}$) that 9H$_2$O is essentially the only observable hydrate under ambient conditions. This large stability difference also suggests that hydration may proceed through stepwise addition of water molecules, with 8H$_2$O as a transient intermediate that rapidly converts to the more stable 9H$_2$O.

From a practical standpoint, the recycling process is \textbf{energy-efficient}, requiring approximately 1.2 MJ/kg compared to 5--10 MJ/kg for virgin synthesis via high-temperature solid-state reaction. This 5--8$\times$ energy savings translates directly to reduced carbon emissions and environmental impact. The process is also \textbf{economically viable}, recovering 90--95\% of material value (\$8--12/kg) while incurring minimal energy costs (\$0.03/kg) and modest processing costs (\$1--2/kg), yielding favorable economics for industrial implementation.

Our thermodynamic framework provides specific optimization guidelines: room temperature hydration with high humidity favors 9H$_2$O formation; dehydration requires elevated temperatures and vacuum $<$10$^{-5}$ bar; particle size reduction accelerates dehydration kinetics; and multi-stage temperature ramping may preserve crystal structure quality. These insights enable rational design of industrial recycling protocols.

Beyond the specific case of Na$_3$SbS$_4$, these results demonstrate that sulfide solid electrolytes are intrinsically recyclable through simple chemical transformations, in contrast to the complex metallurgical processing required for conventional battery materials. This establishes recyclability as an achievable and valuable design criterion for next-generation solid-state battery electrolytes, supporting the development of sustainable battery technologies with closed-loop material flows. The thermodynamic reversibility and energetic favorability of the Na$_3$SbS$_4$ recycling process provides a model system for designing recyclable solid electrolytes in future battery chemistries.

\section*{Acknowledgments}
\small
DFT calculations were performed using the Vienna Ab initio Simulation Package (VASP). Crystal structure for Na$_3$SbS$_4$$\cdot$8H$_2$O was obtained from Tian et al.\ [3], and the Na$_3$SbS$_4$$\cdot$9H$_2$O structure was obtained from the Materials Project database. We thank our experimental collaborators for sharing recycling data and for valuable discussions regarding practical implementation of the recycling process.

\begin{thebibliography}{9}
\small
\itemsep=-2pt

\bibitem{zhang2017}
Z.~Zhang \textit{et al.}, \textit{Inorg. Chem.} \textbf{56}, 3023 (2017).

\bibitem{banerjee2016}
A.~Banerjee \textit{et al.}, \textit{Angew. Chem. Int. Ed.} \textbf{55}, 9634 (2016).

\bibitem{tian2019}
Y.~Tian \textit{et al.}, \textit{Joule} \textbf{3}, 1037 (2019).

\bibitem{kresse1996}
G.~Kresse and J.~Furthmüller, \textit{Phys. Rev. B} \textbf{54}, 11169 (1996).

\bibitem{kresse1999}
G.~Kresse and D.~Joubert, \textit{Phys. Rev. B} \textbf{59}, 1758 (1999).

\bibitem{perdew1996}
J.~P.~Perdew \textit{et al.}, \textit{Phys. Rev. Lett.} \textbf{77}, 3865 (1996).

\bibitem{grimme2010}
S.~Grimme \textit{et al.}, \textit{J. Chem. Phys.} \textbf{132}, 154104 (2010).

\bibitem{blochl1994}
P.~E.~Blöchl, \textit{Phys. Rev. B} \textbf{50}, 17953 (1994).

\bibitem{chase1998}
M.~W.~Chase, \textit{J. Phys. Chem. Ref. Data} Monograph 9 (1998).

\end{thebibliography}

\end{document}
